\documentclass[../main.tex]{subfiles}
\begin{document}
\subsection{Low-dimensional dynamical systems}\label{subsec3.1}
The naive interpretation of a system exhibiting a critical transition is that the temporal evolution $\dot{x}=f(x)$ of a set of observables $x\in \mathbb{R}^{n}$ undergoes a sudden, qualitative change at particular values of an intrinsic (potentially hidden) parameter $\mu\in \mathbb{R}^{m}$. 
Here the dimensionality of the state space $\mathbb{R}^{n}$ is important to characterise as $n \sim \mathcal{O}(1)\ll\infty$ as it fundamentally enables us to apply classical dynamical systems theory for the discovery of equilibria, (hyperbolic) invariant sets and their bifurcation.
In those low-dimensional systems there are multiple causes that may drive the dynamics to tip towards different regimes (two of which we have already informally introduced in the previous Chapter):
\begin{itemize}
     \item bifurcation-induced (B-tipping): also known as dynamic bifurcations, they are the result of dynamical systems passing through a bifurcation of their parameters causing one or multiple attractors to lose their stability;
     \item rate-induced (R-tipping): when the evolution of a system fails to track the time-changing attractor caused by the rate of change of the parameters rather than their values (as opposed to what happens at bifurcation points);
     \item noise-induced (N-tipping): affecting those systems perturbed by sufficiently high noise which causes the state to depart far enough from a neighbour of a stable equilibrium so that it eventually escapes the basin of attraction.
\end{itemize}
In must be emphasized that given any dynamical system, critical transitions may occur because of one or multiple concurrent causes described above and in general it is not possible to discern the original source of such tipping.
% Deterministic transitions: bifurcation-induced (B-)tipping
\subfile{subsubsec3.1.1}
% Different timescales: rate-induced (R-)tipping
\subfile{subsubsec3.1.2}
% Stochastic perturbations: noise-induced (N-)tipping
\subfile{subsubsec3.1.3}
\end{document}
