\documentclass[../main.tex]{subfiles}
\begin{document}
\section{Introduction}\label{sec1}
Complex dynamical systems have found widespread use in the modelling of real-world phenomena given their capability of describing the collective, macroscopic properties of those systems made of a multitude of interactive components.
The analysis of complex systems has thus become particurarly useful for scientific areas such as (s.a.) climate science, ecosystems, biology, sociology and economics.
The non-linear nature of the models makes the rigorous characterisation of these systems notoriously challenging as it enables the emergence of specific properties (self-organising pattern formation and onset of chaos to cite a few) that are shared among very different scientific fields.
Recently however, some specific communities (notably ecology and climate science) have focused their attention on one of those property in particular, that is the existence of abrupt dynamical regimes shifts often referred to as critical transitions or tipping points.
Loosely speaking a complex system, whose time-evolution may be deterministic or stochastic, can undergo a tipping point whenever its state variables abruptly shift away from a stable equilibrium and transition to a new regime that may have catastrophic implications for the properties of the system itself.
It is therefore unsurprising that the reliable and timely prediction of such critical thresholds has attracted much attention from ecologists and climate scientists in the past two decades with disrupting phenomena noticed experimentally in a range of catastrophic events in the past and that are likely to happen in the near future as well.
This report concerns the investigation of indicators of critical transitions in complex dynamical systems, i.e. the identification of robust, measurable and detectable quantities that can provide early-warning signals (EWS) of incoming catastrophic tipping events.
\paragraph{Structure}
This document is organised in $4$ Chapters detailing and motivating the research of EWS. 
The content is structured as follows: in the present (first) Chapter we will outline the incentives for the study and analysis of these precursors (Section \ref{subsec1.1}), as provided by empirical evidence from the natural world, followed by a detailed review of the history of EWS proposed in the last $25$ years (Section \ref{subsec1.2}), their mathematical foundation and the scientific disciplines in which they have been succesfully detected; 
in the second Chapter we will review the basic, fundamental theories upon which the research of catastrophic events is based on, namely differential equations (Sections \ref{subsec2.1} and \ref{subsec2.2}) and stochastic calculus (Section \ref{subsec2.3});
in the third Chapter we focus on the construction of a mathematical framework for critical transitions and their EWS in low-dimensional (Section \ref{subsec3.1}) and high-dimensional (Section \ref{subsec3.2}) systems; 
this will motivate the research itself reported in the following Chapters, in particular in the fourth Chapter we address limitations of current methods in finding precursors in terms of lack or universality given their \textit{top-down} nature;
Chapter 5 concerns the investigation of spatially-extended systems and the rich dynamics that allows the formation of patterned instabilities.
% Motivation
\subfile{subsec1.1}
% Research context
\subfile{subsec1.2}
\end{document}
