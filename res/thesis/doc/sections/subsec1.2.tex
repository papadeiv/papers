\documentclass[../main.tex]{subfiles}
\begin{document}
\subsection{Research context}\label{subsec1.2}
When considering EWS of critical transitions a distinction has to be made between the observed precursor, that is a phenomoenological pattern that the system exhibits as it approaches the threshold, and the indicator itself, which is a specific measurable signal that characterises the registered phenomenon.
This discernment is fundamental in the characterisation of EWS given by the fact that the same precursor can be measured by different indicators (see Table \ref{tab2.1}).
As an example consider \textit{critical slowing down} (CSD), arguably the most studied and well understood precursor in climate and ecological systems. 
It specifically refers to the observed behaviour of a dynamical system to become less resilient to perturbations of its stable equilibrium as a critical transition is approached. 
In other terms if the state of the system tracking an attractor comes close to a bifurcation then CSD is the symptomatic slower recovery towards the attractor w.r.t. small (noisy) perturbations.
This phenomenon can be measured in a variety of ways and thus a number of different indicators (both analytical and statistical) have been proposed throughout the years.
Other forms of precursors are \textit{flickering} and \textit{pattern-formation} and they also have been associated to the detection of multiple signals.
Some questions naturally arise following this separation, namely is there a better indicator than others for the clear and early detection of a critical transition? 
Is such indicator robust i.e. can it provide consistently a signal for the given precursor observed from different systems? 
Does it exist a universal measure that characterises critical transitions a-priori i.e. without prior knowledge of neither the presence of a bifurcation nor the functional model of reference?
Answers to these queries have been the subject of several efforts in the last 10 years and, to some extent, generic and robust indicators of abrupt regimes shifts have been achieved for a subset of simplified models.
For the rest of the present section we will outline a condensed but representative genealogy of EWS of critical transitions.
% Critical slowing down as loss of resilience
\subfile{subsubsec1.2.1}
% Criticism of EWS from climate and ecosystem timeseries
\subfile{subsubsec1.2.2}
% Flickering as an alternative sign of critical tresholds
\subfile{subsubsec1.2.3}
% Investigation of spatially extended systems
\subfile{subsubsec1.2.4}
\end{document}
