\documentclass[../main.tex]{subfiles}
\begin{document}
\subsection{Parameter shift}\label{subsec:shift}

With Definition \ref{def:r_tipping} we know that we must model the non-autonomous part of the system in such a way that an initial condition $x_{0}$, which at time $-T$ starts in the basin of $x_{1}$ (or $x_{2}$), when propagated forward in time will have $x_{2}$ (or $x_{1}$) as its $\omega-$limit set.
In other words we want the trajectory $x_{0}(t)$ to somehow cross the boundary of the basin of attraction at some finite time $t$, which eventually is the cause of the R-tipping.
With Observation \ref{obs:r_tipping_replicator} we know that such boundary for the replicator equation is given by $x_{*}$ which is unstable for all $(\alpha,\beta)$ in $\tilde{\Gamma}_{*}$.
Therefore one way to reproduce R-tipping is to make either (or both) the parameters in $\tilde{\Gamma}_{*}$ to change over time as this will \textit{move} $x_{*}$ in $\Omega$ and consequentally also change the boundary of the $2$ basins of attraction.

% Smooth monotonic ramp 
\subfile{subsubsec:ramp}
% Modelling the shift 
\subfile{subsubsec:modelling}

\end{document}
