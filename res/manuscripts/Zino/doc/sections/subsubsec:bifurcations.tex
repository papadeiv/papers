\documentclass[../main.tex]{subfiles}
\begin{document}
\subsubsection{Bifurcation structure}\label{subsubsec:bifurcations}

Following Lemma \ref{lemma:stability} we can easily identify local codim$-1$ bifurcations of the replicator system as per the following result.

\begin{theorem}[label=thm:bifurcations]{}{}
     Let the conditions of Lemma \ref{lemma:stability} be satisfied, then the subsets $\alpha=0$ and $\beta=0$ of the admissable set $\Gamma$ are locii of transcritical bifurcations for \eqref{eq:replicator_reduced}. 
\end{theorem}

\begin{proof}
     Given a scalar (i.e. $1-$dimensional) dynamical system $f(x;\mu)$ in state variable $x\in \mathbb{R}$ and (fixed) parameter value $\mu\in \mathbb{R}$ the conditions for a transcritical bifurcation of an equilibrium $x_{0}$ at parameter value $\mu_{0}$ are the following
     \begin{enumerate}
             \item $(D_{x}f)(x=x_{0};\mu=\mu_{0}) = 0$;
             \item $(D_{\mu}f)(x=x_{0};\mu=\mu_{0}) = 0$;
             \item $(D_{xx}f)(x=x_{0};\mu=\mu_{0}) \neq 0$;
             \item $\bigg((D_{\mu x}f) - (D_{xx}f)(D_{\mu\mu}f)\bigg)(x=x_{0};\mu=\mu_{0}) > 0$;
     \end{enumerate}
     The linearised dynamics of \eqref{eq:replicator_reduced} reads
     \begin{equation*}
             (D_{x}f)(x) = -3(\alpha + \beta)x^{2} + 2(\alpha + 2 \beta)x - \beta\,, 
     \end{equation*}
     which evaluated at the $3$ equilibria yields
     \begin{align*}
             (D_{x}f)(x=x_{1}) =& -\beta\,, \\
             (D_{x}f)(x=x_{2}) =& -\alpha\,, \\
             (D_{x}f)(x=x_{*}) =& -\alpha(3 \alpha^{2} + \alpha(3 \beta + 4) + 2 \beta + 1)\,.
     \end{align*}
     It is straightforward from the above to realise that $x_{1}$ approaches loss of hyperbolicity from a state of stable attraction as $\beta\nearrow0$ and, similarly, so does $x_{2}$ as $\alpha\nearrow0$.
     Furthermore, from \eqref{eq:unstable_eq} it is also straightforward to see that
     \begin{equation*}
          x_{*}\to x_{1}\,,\;\beta\to0\quad \text{and} \quad x_{*}\to x_{2} \,,\;\alpha\to0\,.
     \end{equation*}
     To prove that $\alpha=0$ and $\beta=0$ are transcritical bifurcations for $x_{*}$ we essentially need to verify that the conditions $1.-4.$ listed above hold for $x=x_{*}$ and $\{\alpha,\beta\}\ni\mu=0$.
     This can be done by direct calculation for $\alpha=0$ and $\beta=0$ individually by fixing one of the two parameters and letting $\mu$ being the bifurcating one.
     We omit such calculations which, alebit trivial, are long and tedious to detail here and remark that this is a standard procedure in classical bifurcation theory (see e.g. \cite{Glendinning94}).
\end{proof}

\begin{observation}\label{obs:separatrices}
     From Definition \ref{def:restricted_admissable_set} it follows trivially that $\newprime{\Gamma}$ and $\Gamma_{*}$ are disjoint except for a subset of measure $1$ in $\mathbb{R}^{2}$ that correspond to the union of the locii of transcritical bifurcations $\alpha = 0$ and $\beta = 0$.
\end{observation}

The meaning of \ref{obs:separatrices} is that the locii of transcritical bifurcations in $\mathbb{R}^{2}$ act as separatrices in the bifurcation set of \eqref{eq:replicator_reduced} and as a result separate the interior of $\newprime{\Gamma}$ from the interior of the restricted admissable set $\Gamma_{*}$.

% Bifurcation set 
\subfile{par:bif_set}

\end{document}
