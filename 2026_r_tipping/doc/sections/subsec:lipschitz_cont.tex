\documentclass[../main.tex]{subfiles}
\begin{document}
\subsection{Lipschitz continuity}\label{subsec:lipschitz_cont}

\begin{definition}[Lipschitz function]\label{def:lipschitz_fun}
        A continuous and differentiable function $f:\mathbb{R}\to \mathbb{R}$ is Lipschitz is there is a $L>0$ s.t.
        \begin{equation*}
                |f(x)-f(y)| < L\,|x-y|\,,\quad\forall x\in \mathbb{R}\setminus\{y\}\,.
        \end{equation*}
\end{definition}

\begin{theorem}[label=thm:mean_value_thm]{Mean Value Theorem}{}
        If $f$ is differentiable then $\forall x,y\in \mathbb{R}$, $x\neq y$ there exist $z\in(x,y)$ s.t.
        \begin{equation*}
                \frac{f(x)-f(y)}{x-y} = \newprime{f}(z)\,.
        \end{equation*}
\end{theorem}

\begin{corollary}
     By Definition \ref{def:lipschitz_fun} and Theorem \ref{thm:mean_value_thm}, if there exists $L>0$ s.t. $|\newprime{f}(z)|<L$, $\forall z \in \mathbb{R}$ then $f$ is Lipschitz. This further implies that any $f$ whose first derivative $\newprime{f}$ is bounded is necessarly Lipschitz.
\end{corollary}

% Computing Lipschitz constant 
\subfile{subsubsec:lipschitz_const}

\end{document}
