\documentclass[../main.tex]{subfiles}
\begin{document}
\subsubsection{The state variables}\label{subsubsec:state_variables}

The dynamical system introduced by Galor and Weil (GW) is $4-$dimensional.
However, given its complexity, sever auxillary quantitites are introduced which do not update according to a rule that couples to the dynamical system itself; we will call these quantities \textit{observables}.
We will first provide a descriptive outline of the macroeconomic meaning of all the quantities involved and then separate them into state variables and observables.
The model consists of agents in overlapping generations meaning that at a given iteration $t$ the agents participate to the economy both as childern and adults (here a time increment $t$ denotes the passage of one generation, approximately spanning 20 terrestrial years).
The per-capita income is denoted by 
\begin{equation}\label{eq:income}
     z_{t} = h_{t}^{\alpha}x_{t}^{1-\alpha}\,,
\end{equation}
where $h_{t}$ is human capital, $x_{t}$ is resources available to the agent and $\alpha$ is the share of labour in the production of goods.
Introducing the total population of adults $L_{t}$, the total land $X$ and the level of technology $A_{t}$ we can further characterise \eqref{eq:income} by noticing that
\begin{equation}\label{eq:income_explicit}
     x_{t} = \frac{A_{t}X}{L_{t}} \;\Rightarrow\; z_{t} = h_{t}^{\alpha}\bigg(\frac{A_{t}}{L_{t}}\bigg)^{1 - \alpha} \,,
\end{equation}
where we explicitly set $X=1$ for simplicity.
Also important in GW's model is the relative increase in technology from one generation to the next
\begin{equation}\label{eq:relative_tech_increase}
     g_{t+1} = \frac{A_{t+1} - A_{t}}{A_{t}}\,.
\end{equation}
Finally, we denotes with $e_{t+1}$ the education invested by an adult at generation $t$ towards a child that becomes an adult at generation $t+1$.

The state variables of our dynamical systems are $X_{t} = (g_{t},\,A_{t},\,e_{t},\,L_{t})$ with $x_{t},\,h_{t},\,z_{t}$ being observables. More explicitly
\begin{equation}\label{eq:galor_system}
        \underbrace{
                \begin{cases}
                        g_{t+1} = \mathcal{G}(e,\,L)\,, \\
                        A_{t+1} = \mathcal{A}(A,\,e,\,L) = (1 + \mathcal{G}(e,\,L))A\,, \\ 
                        e_{t+1} = \mathcal{E}(e,\,L)\,, \\ 
                        L_{t+1} = \mathcal{L}(g,\,A,\,e,\,L) = 
                           \begin{cases}
                               \frac{\gamma}{\tau + \mathcal{E}(e,\,L)}\,,\;\;z_{t}\geq \frac{1}{1-\gamma},\, \\
                               \frac{1 - \frac{1}{z_{t}}}{\tau + \mathcal{E}(e,\,L)}\,,\;\;1 < z_{t} < \frac{1}{1 - \gamma},\, \\
                               0\,,\;\;z_{t}\leq 1.
                           \end{cases}
                \end{cases}
        }_{\text{state variables}}
        \,,\quad
        \underbrace{
                \begin{cases}
                        x_{t} = x(A,\,L) = \frac{A}{L}\,, \\
                        h_{t} = h(g,\,e)\,,\\ 
                        z_{t} = z(g,\,A,\,e,\,L) = h^{\alpha}(g,\,e)\,x^{1-\alpha}(A,\,L)\,,
                \end{cases}
        }_{\text{observables}}
\end{equation}
with $\gamma\in(0,1)$ and $\tau > 0$ (unit time cost of education).

\end{document}
