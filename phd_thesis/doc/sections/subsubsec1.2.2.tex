\documentclass[../main.tex]{subfiles}
\begin{document}
\subsubsection{Criticism of EWS from climate and ecosystem timeseries}\label{subsubsec1.2.2}
A substantial wealth of works was published in the following decade marking the beginning of intense research and categorisation of EWS from a dynamic perspective. 
We can think of the years spanning from 2007 to 2015 as the \textit{gold rush} era of EWS in climate and ecology.
\paragraph{Late 2000s}
Regarding the former, in 2007 a method based on detrended fluctuation analysis (DFA) was introduced by Livina and Lenton \cite{Livina07} to improve upon the estimation of the decay rate, and hence the proximity to a climate bifurcation, from paleoclimate records. 
They realised that the non-stationarity of (climate) timeseries approaching a bifurcation somehow pollute the statistical scaling properties used to estimate the non-linear shifts. 
The importance of detrending dynamical timeseries will be made clear in the third Chapter of the present work.
In the same year van Nes and Scheffer \cite{vanNes07} took note of this and explored the estimation of the recovery rate from perturbations by fitting an exponential process to the timeseries generated by $6$ simple, $1-$dimensional models of ecosystems. 
They also curcially mentioned the potential detection of \textit{false positives} and \textit{false negatives}.
This crossover between the two communities eventually led to the 2008 pivotal work by Dakos, Scheffer, van Nes et al. \cite{Dakos08} in which the first coefficient of a fitted autoregressive AR(1) model (which measures the autocorrelation at lag-1) was shown to consistently predict the onset of critical transitions in $8$ major paleoclimate events ($3$ of which are reported in Figures \ref{fig1.1}(a)-\ref{fig1.1}(c) above), i.e. using real-world data rather than model-based simulations.
In the following 3 years several forms of indicators prognostic of CSD were proposed and applied to different population models. 
We mention, among others: kurtosis and skewness \cite{Biggs09} used in a fisheries food web model (which found weaker signals compared to AR(1) and reddening of the spectral density ratio); the analytical recovery rate of Hopf and transcritical bifurcations of the $2-$dimensional predatory-prey and Lotka-Volterra models \cite{Chisholm09}; variance, autocorrelation and skewness in the transcritical bifurcation of an experimental population in a deteriorating environment \cite{Drake10}; variance, return rate, skewness and spectral ratio in a monitored lake ecosystem \cite{Carpenter11a}; the total variance of a fitted drift-diffusion-jump (DDJ) model \cite{Carpenter11b} for non-parametric, data-generating processes.
Soon after reviews of EWS of critical transitions became available, most notably Scheffer, Carpenter, Dakos et al. \cite{Scheffer09} and Lenton \cite{Lenton11}. In 2009 Scheffer also published a book \cite{Scheffer09b} on critical transitions compiled until then in natural and human sciences.
\paragraph{Early 2010s}
With this new discipline gaining traction it also, naturally, attracted some criticism. 
In 1977 the implications of catastrophic events in biological and social sciences was put in question \cite{Zahler77} by \textit{``incorrect reasoning"} and \textit{``far-fetched assumptions"} regarding cusp bifurcations.
With regards to the robustness of the statistical indicators a more recent work in 2010 \cite{Ditlevsen10} argued that not all paleoclimate records show critical transitions due to bifurcation and are rather caused by stochastic fluctuations (known today as N-tippings, see Section \ref{subsec2.1}) which have very limited predictibility. 
The authors noted that by assessing the statistical significance of the variance and autocorrelation of records of the Dansgaard-Oeschger events (analysed by the ice core data in the North Greenland ice core project, or NGRIP) one showed a monotonic trend while the other showed no signal. 
This, the authors argued, was evidence that no bifurcation was approached by the system which instead tipped over to an alternative stable regime driven by noise alone, which is contrary to what was claimed in previous works \cite{Dakos08,Scheffer09}.
In addition, the same year, another work showed that abrupt population shifts in particular ecosystems (specifically when the magnitude of the stochastic fluctuations increases, i.e. higher noise levels) occured without forewarning \cite{Hastings10} of CSD (i.e. false negatives or \textit{missed alarms}).
The mathematical basis of the authors was that all the previous works showed EWS for (simplified) systems that have a smooth dependence of the potential landscape on the underlying parameter and did not consider real, complex systems exhibiting, for example, chaotic attractors. 
Ensemble simultations of the Ricker map did in fact lead to a failure of the aforementioned EWS in detecting the population collapse.
Increasing evidence supporting the lack of consistency of the proposed indicators was further provided in 2013 by Dakos, Scheffer et al. \cite{Kefi13} and others \cite{Boerlijst13}. 
In particular the latter followed the examples laid out in the previous papers, i.e. providing evidence of \textit{silent catastrophes} while the former went in the opposite direction and reported that variance and autocorrelation give a trend even for non-catastrophic events (i.e. false positives).
These $4$ works pointed out the urgency of: 
\begin{enumerate}
     \item a robust characterisation (in terms of statistical significance and some form of universality across different complex systems) of EWS;
     \item a deeper (and desirably rigorous) understanding of the mathematical structure of the analysis motivating the existence of EWS.
\end{enumerate}
The second of those two needs begun to be investigated by Kuehn in 2011 \cite{Kuehn11} and we discuss it in the next Chapter (see Sections \ref{subsubsec2.1.1} and \ref{subsubsec2.1.3}). The questions regarding the robustness and universality of the leading indicators of CSD also started to be addressed shortly thereafter and we will explore them further ahead (see Sections \ref{subsec2.3} and \ref{subsec3.1}).
\end{document}
