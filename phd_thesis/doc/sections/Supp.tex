\documentclass[../main.tex]{subfiles}
\begin{document}
\section{Supplementary material}\label{Supp}
\subsection{Data sources for real-world timeseries in Figure 1}\label{S1}
We hereby list all the sources from which the historical data used for the timeseries of Figure \ref{fig1.1}. The \textbf{end of Greenhouse Earth} (a) timeseries is a collection of irregularly spaced samples from tropical Pacific sediment core records for the Eocene transition from Greenhouse to Icehouse Earth's climate \cite{Tripati05}; 
the data itself was downloaded from the World Data Center for Paleoclimatology\footnote[1]{https://www.ncei.noaa.gov/products/paleoclimatology} hosted by the National Oceanic and Atmospheric Administration (NOAA).
The \textbf{end of the last glaciation} (b) represents the concentration of Deuterium found in Vostok's Antarctic ice core \cite{Petit01} from which temperature reconstruction for the past $420,000$ years was performed; the dataset was downloaded from the aforementioned Data Center\footnotemark[1].
The \textbf{desertification of North Africa} (c), coinciding with the abrupt end of the African humid period, is depicted as a measurement of the mean Sea Surface Temperature (SST) reconstructed from the Ocean Drilling Program (ODP) site 658C \cite{deMonecal00}; the data is sourced from the georeferenced database PANGAEA\footnote[2]{https://www.pangaea.de/}.
The \textbf{yeast population collapse} (d) data comes from a controlled experiment monitoring the density of budding yeast cells at steady dilution factors of sucrose \cite{Dai12}; these series of experiments were then used to form hypothesis on the role of CSD (and its indicators) preceeding systemic collapse in ecosystems \cite{Proverbio23}.
The \textbf{heavy Fermi compound phase transition} (e) experimental datapoints are related to the Fermionic breakdown shown as a sharp decrease in the (THz) spectroscopy resonance of YbRh$_2$Si$_2$ at constant magnetic field (130 mT) and varying temperature; these results were added as supplementary data in \cite{Yang23}.
The \textbf{2009 global financial crisis} (f) is reflected, among other indicators, as a sudden collapse of the historical S\&P 500 stock market index and recent studies \cite{Dmitriev17,Diks18} suggest a CSD for this and other market crashes \cite{Vandewalle98}; the timeseries is collected from the Federal Reserve Bank of St. Louis \footnote[3]{https://fred.stlouisfed.org/series/SP500}.
\subsection{Potential of the bistable saddle-node bifurcation in Figure 3}\label{S2}
We consider an additive noise Ito process in potential form
\begin{equation*}
     dx = -U^{'}(x;\mu)dt + \sigma dW\,,
\end{equation*}
where we set $U(x;\mu)=\mu x + x^2 - x^3 + \frac{1}{5}x^4$ to be the primitive dynamics. The system has two saddle-node bifurcations at $\mu\approx-0.38$ and $\mu\approx1.62$. The two ensembles simulated in (a) and (b) have a total of $100$ independent trajectories each with slow, linear ramping of the parameter with time $\varepsilon=10^{-2}$ and noise levels $\sigma = 0.1$ and $\sigma = 1.2$ respectively.
\end{document}
