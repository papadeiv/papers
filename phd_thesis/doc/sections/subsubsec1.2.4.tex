\documentclass[../main.tex]{subfiles}
\begin{document}
\subsubsection{Investigation of spatially extended systems}\label{subsubsec1.2.4}
The first evidence of accounting spatial information in the context of critical transitions is from the 1998 series of works by Gandhi and colleagues \cite{Gandhi98} which realised that, whenever the dynamics of competing species are dominated by spatial clusters (as opposed to the usual assumption of well-mixing of the population densities) proxies of CSD marking the distance to extinction can be determined s.a. the correlation length (average cluster size). 
These works also noted that spatial aggregation through mean-field approximation (MFA) averages out the fluctuations that become dominant as the phase transition is approached thereby providing some insights on the loss of information when space is not accounted in dynamical models.
\paragraph{2000s}
Some years later in 2004, the MFA of well-mixed ecosystems with bistable regimes was again called into question when considering natural catastrophic shifts that feature the emergence of self-organising patches of consumers and resources \cite{Rietkerk04}, s.a. those found in arid and savanna vegetation patterns. 
This called for the extension to spatial domains to recognise self-organised patchiness as a precursor of regime shifts.
One year later, self-organised patchiness (equivalently known as pattern-formation or Turing instability) was again suggested as a forerunner of catastrophic shifts for asthma attacks \cite{Venegas05}. 
It compared experimental evidence via positron emission tomography (PET) of cluster ventilation defects (CVDs) in the lung with the effect of spatial heterogeneity breaking the uniform smooth muscle activation in a computational model of a symmetric bronchial tree. 
It was observed specifically how the local bistability of the airway tree induces the patchy self-organisation of CVDs. 
Further evidence of spatial patterns preluding ecosystem collapses was presented in 2007 when it was observed that the size distribution of the vegetation patches in Mediterranean arid environments (found in Souther Spain, Greece and North-Western Morocco) follows a power law with the number of such patches prior to desertification \cite{Kefi07}. 
As it became increasingly clear that pattern-formation was yet another form of precursor capturing the insofar neglected spatial information, a consistent statistical measure of such a phenomenon was still lacking.
In 2009 Guttal et al. \cite{Guttal09} put a remedy to this deficiency by introducing spatial variance and spatial skewness as a first form of leading indicators. 
In particular they show how an increase in the former and changes in the latter provide clear indications of pattern-forming transitions to bistablity regimes in a spatially extended population model.
Furthermore it was demonstrated how these two indicators can overcome the limitations imposed by low temporal resolution of the timeseries of the MFA of the model.
\paragraph{Mid 2010s}
The 2010s thus marked the beginning of the analysis of spatio-temporal EWS, particularly by the ecologists.
Carpenter and Brock \cite{Carpenter10} used the discrete Fourier transform (DFT) to extend the previously conceived concept of spectral reddening of a timeseries experiencing CSD \cite{Kleinen03} to shifts to lower spatial frequencies for period-doubling (Ricker map) and bistable (predator-prey and harvester-prey) transitions in $4$ ecological models.
Concurrently Dakos, van Nes, Scheffer et al. \cite{Dakos10} proposed spatial correlation as the leading indicator of CSD prior to the bifurcations in overharvesting, eutrophication and vegetation turbidity reaction-diffusion models.
Specifically they compared the signal provided by Moran's two-points correlation coefficient (used to calculate the spatial correlation of neighboring cells in a $2-$dimensional lattice) with that of the temporal autocorrelation at lag$-1$ of the spatially aggregated model and found that the former consistently outperforms the latter.
In 2011 the aforementioned authors and others \cite{Dakos11} applied the same ideas to spatially-patterned vegetation changes at the onset of desertification and concluded that CSD was not detected by rising temporal variance and autocorrelation as opposed to their spatial counterparts which were able to show positive trends leading to both CSD and Turing instabilities.
Experimental evidence from a whole-ecosystem food-web monitoring of predators' distribution in a lake \cite{Cline14} also indicated that spatial variance and shifts to low spatial frequencies using the DFT provided EWS for the emergence of prey fish patchiness one year in advance of the bistable transition in the populations.
\paragraph{Late 2010s and 2020s}
Nevertheless, pattern-formation is only one of the many dynamical phenomena that affect the stability of spatio-temporal systems (as we will explore in the third Chapter of the present work). 
In fact Kuehn et al. in 2013 \cite{Kuehn13} and Leemput, van Nes and Scheffer in 2015 \cite{Leemput15} indipendently considered travelling waves of invasion fronts connecting two well-separated initial populations distributions. They both indicated the slow-down (or \textit{pinning}) of the wavespeed as a precursor to the critical point at which the resilience of two alternative stable states of high and low biomass is equal (called Maxwell point).
Investigation of spatio-temporal EWS also extended to nework-based analysis in the later part of the decade with the maximum element of the covariance matrix \cite{Suweis14}, the noise-dependent characteristic return time of a dimensionality reduction of a coupled network \cite{Liang17}, the spatial correlation of a multiplex disease network \cite{Jentsch18} and the slower reovery from perturbation of mutualistic communities \cite{Lever20} all providing candidate spatial indicators.
The latest efforts of the current decade also proposed the adoption of techniques from statistical mechanics \cite{Chen19,Gottwald20,Donovan22} for spatio-temporal systems as well as deep-learning \cite{Bury21,Dylewsky23} for the automatic detection of combined signals albeit with questionable insights gained from the interpretation of their meaning.
\end{document}
