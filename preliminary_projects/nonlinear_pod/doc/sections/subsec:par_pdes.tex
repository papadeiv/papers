\documentclass[../main.tex]{subfiles}
\begin{document}
\subsection{Parametrised PDEs}\label{subsec:par_pdes}

Parametrized PDEs arise in many scientific and engineering problems whose solutions depend on parameters representing physical properties, boundary conditions or geometric configurations. 
In the following, without loss of generality, we consider a parameter $\mu\in \mathcal{P}\equiv \mathbb{R}$ and a parametriesed evolution PDE described as an initial boundary value problem (IBVP)
\begin{equation}\label{eq:ibvp}
   \begin{cases}
           \partial_{t}u + \mathcal{L}_{\mu}u = f(x,u,\mu)\,, \quad x\in\mathring{\Omega}\,,\;t\in(0,T]\,,\\
       u(x,t) = g(x)\,,\quad x\in\partial\Omega\,,\;t\in(0,T]\,,\\ 
       u(x,0) = u_{0}(x)\,,\quad x \in\mathring{\Omega}\,,
   \end{cases}
\end{equation}
where the specified Dirichlet boundary conditions (BCs) and a certain initial condition (IC).
Notice that \eqref{eq:ibvp} is autonomous in the sense that neither the PDE nor the BCs depend explicitly on time.
In an engineering setting \eqref{eq:ibvp} models some physical behaviour that is parametrised by $\mu$.
A canonical example is for \eqref{eq:ibvp} to be incompressible Navier-Stokes and $\mu$ to be the viscosity $\nu$ of a given fluid.

\end{document}
