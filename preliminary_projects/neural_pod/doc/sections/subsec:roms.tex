\documentclass[../main.tex]{subfiles}
\begin{document}
\subsection{Reduced order methods (ROMs)}\label{subsec:roms}

In engineering design workflows it is often the case that one requires to understand the parameter dependence on the solution of \eqref{eq:ibvp}.
An immediate example that follows from the Navier-Stokes case introduced above is that an aeronautics engineer would like to optimise the design of an airfoil to maximise the lift (given by the pressure field solving the Navier-Stokes equations) in a range of altitudes in the atmosphere (making the density of the fluid $\rho$ the parameter of the PDE).
Since no analytical solution is available to the engineer, numerical simulations are necessary for the design.
As the optimisation process requires solving the PDE for a range of densities, the engineer would need to run a numerical simulation at each of the desired parameter values.
This procedure is computationally very expensive and inefficient since each single run could take several hours to complete (depending on the complexity of the model).
A ROM is a numerical technique whose aim is to approximate the solution space (i.e. a manifold of numerical solutions of \eqref{eq:ibvp} with parameters in $\mathcal{P}$) with low-dimensional subspaces, enabling fast and accurate evaluations for an arbitrary number of parameter values.

% Classification 
\subfile{subsubsec:classification}

\end{document}
