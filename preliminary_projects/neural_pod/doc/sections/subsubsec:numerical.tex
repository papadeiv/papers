\documentclass[../main.tex]{subfiles}
\begin{document}
\subsubsection{High-order numerical solutions}\label{subsubsec:numerical}

The abstract procedure of a numerical method for an IBVP is to find a finite-dimensional subspace $V_{h}$ of the functional set $V$ (usually a Banach or Hilbert space) where the solution $u$ of the PDE lives.
This is usually achieved by converting the differential problem into an algebraic one by means of operator discretisation.
There are several numerical schemes to achieve this purpose of which we mention, in descending order of rigorousness, finite elements (FEs), finite volumes (FVs) and finite differences (FDs).
Despite the choosen method, the end result is common: generate a high-dimensional linear system of equations in $\mathbb{R}_{N_{h}}$ whose solution $\boldsymbol{u}_{h}$ is a high-fidelity approximation of $u\in V$.
In the following we briefly describe the main steps of a generic Galerkin method in which the approximate solution $\boldsymbol{u}_{h}$ sought in $\mathbb{R}^{N_{h}}$ is optimal in the sense that minimises the distance with respect to (w.r.t.) the functional space $V$.
This task is pursued by requiring that the residual of this finite-dimensional approximation is orthogonal to the subspace $\mathbb{R}^{N_{h}}$ (known as \textit{Galerkin orthogonality}), i.e. its inner product w.r.t. each of the basis vectors of the subspace must yield $0$.
We outline the specific steps of the approximation schemes mentioned above in the Appendix \ref{sec:appendix}.

% Weak/variational form
\subfile{par:weak_form}
% Finite-dimensional linear form 
\subfile{par:linear_form}

\end{document}
