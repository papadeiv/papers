\documentclass[../main.tex]{subfiles}
\begin{document}
\subsubsection{Smooth monotonic ramp}\label{subsubsec:ramp}

To analyse and prove rate-induced tipping in \eqref{eq:nonautonomous_shifted} we will refer to the rigorous framework introduced in \cite{Ashwin17}.
In it a non-autonomous dynamical system that presents R-tipping requires a $C^{2}-$smooth and bounded parameter shift $\Lambda(t)$ which is not necessarly monotonic.
The parameter shift $\Lambda(t)$ shall connect two parameter values $\lambda_{-},\,\lambda_{+}$ in the time asymptotic regime, i.e.

\begin{equation}\label{eq:generic_shift}
     \text{range}(\Lambda(t))=(\lambda_{-},\lambda_{+})\quad \text{and}\quad \lim_{t\to\pm\infty}\Lambda(t)=\lambda_{\pm}\,.
\end{equation}

Furthermore we ask for our parameter shift to be quasi-stationary at almost every time with the exception of a small subset $(t_{a},t_{b})$.
In other words we ask
\begin{equation}\label{eq:generic_shift_derivative}
        \dot{\Lambda}(t) \approx 0\,,\;\forall t\in \mathbb{R}\setminus(t_{a},t_{b})\,. 
\end{equation}

Conditions \eqref{eq:generic_shift}-\eqref{eq:generic_shift_derivative} are necessary to ensure that the system has a stable attractor in both the past and future limits and that such attractor does not drift too fast once the transient regime is ended or before it started.
Furthermore, even if the parameter shift does not need to be monotonic (as per \cite{Ashwin17}), we will impose this further constraint for simplicity.

\end{document}
