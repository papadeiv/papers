\documentclass[../main.tex]{subfiles}
\begin{document}
\section{Non-autonomous augmentation}\label{sec:nonautonomous_augmentation}

In order ro reproduce rate-induce tipping (R-tipping) for a non-autonomous formulation of \eqref{eq:replicator_reduced} we need the phase space to contain at least two stable equilibria.
This restrict ourselves to the subset $\tilde{\Gamma}_{*}:=\mathbb{R}^{+}\times \mathbb{R}^{+}\subset\Gamma_{*}$ of the restricted admissable set $\Gamma_{*}$ which corresponds to the regime of a coordination game as per Lemma \ref{lemma:stability} and as depicted in Figure \ref{fig:autonomous}.

\begin{definition}[Irreversible R-tipping]\label{def:r_tipping}
        Let $\mathbb{B}(x_{j},t)$ be the basin of attraction of the stable equilibrium $x_{j}$, $j\in\mathbb{N}$, at time $t\in \mathbb{R}$, $T>0$ denoting a time instant and $x_{0} := x(-T) \in \mathring{\mathbb{B}}(x_{j},-T)$ being an initial condition that a time $-T$ lies within the interior of the basin of $x_{j}$. 
        Then we say that the forward solution $x_{0}(t):=\phi_{t}(x_{0})$ has undergone irreversible R-tipping if $x_{0}(t)\to x_{k}$, $t\to+\infty$, with $k\neq j$.
\end{definition}

With the above we specify the condition for which R-tipping can be observed for a non-autonomous augmentation of the replicator system in a regime of coordination game.

\begin{observation}\label{obs:r_tipping_replicator}
        In the case of \eqref{eq:replicator_reduced} then $j\in\{1,2\}$, $\forall(\alpha,\beta)\in\tilde{\Gamma}_{*}$.
        The basins $\mathbb{B}(x_{1/2},t)$ are therefore uniquely identified, at each time instant $t\in \mathbb{R}$, by the location of $x_{*}$ in $\Omega$.
        More precisely
        \begin{align*}
                \mathbb{B}(x_{1},t) =&\, [x_{1}=0,x_{3})\,, \\
                \mathbb{B}(x_{2},t) =&\, (x_{3},x_{2}=1]\,. \\
        \end{align*}
        Furthermore $\Omega = \mathbb{B}(x_{1},t)\cup x_{*} \cup\mathbb{B}(x_{2},t)$, $\forall t\in \mathbb{R}$.
\end{observation}

We now have the tools to formulate the non-autonomous replicator equation in a way that it guarantees the necessary conditions for irreversible R-tipping.

% Parameter shift 
\subfile{subsec:shift}
% Numerical results 
\subfile{subsec:results}

\end{document}
