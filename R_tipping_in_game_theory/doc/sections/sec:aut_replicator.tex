\documentclass[../main.tex]{subfiles}
\begin{document}
\section{Autonomous replicator equation}\label{sec:aut_replicator}
We consider the following $1-$dimensional dynamical system

\begin{equation}\label{eq:replicator_implicit}
        \dot{x} = f(x) = x(1-x)(r_{1}(x) - r_{2}(x))\,,
\end{equation}

where $r_{1}(x) = ax + b(1-x)$, $r_{2}(x) = cx + d(1-x)$ and $a,b,c,d\in \mathbb{R}$.
Rearranging the terms of the last factor in \eqref{eq:replicator_implicit} yields
\begin{align}
        \dot{x} =& x(1-x)(ax + b(1-x) - cx - d(1-x)) = \nonumber \\
                =&  x(1-x)((a-c)x - (d-b)(1-x)) \,. \label{eq:replicator_full}
\end{align}
This system's parameter space is therefore $\mathbb{R}^{4}$.
We can define $\mathbb{R}\ni \alpha:=a-c$ and $\mathbb{R}\ni \beta:=d-b$ so that we can write \eqref{eq:replicator_full} in a compactified form
\begin{equation}\label{eq:replicator_reduced}
        \dot{x} = x(1-x)(\alpha x - \beta(1-x))\,.
\end{equation}
In the following we will use \eqref{eq:replicator_full} or \eqref{eq:replicator_reduced} depending on the conditions we want to derive.

% Game-theoretic setting 
\subfile{subsec:aut_setting}

\end{document}
