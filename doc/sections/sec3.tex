\documentclass[../main.tex]{subfiles}
\begin{document}
\section{Critical transitions}\label{sec3}
In the first Chapter we explored how several different EWS have been derived for three different forms of precursors of critical transitions in natural phenomena.
In most of the cited literature the data from which these signals arose came from idealised mathematical models.
This is particularly the case for ecological and climate systems where meaningful real-life data gathering is complex and requires a substantial undertaking in terms of resources.
As a result of this idealisation the quality and robustness of a proposed EWS heavily relies on the accuracy with which such simplified models capture the essence of a phenomenon of interest.
To that end in this Chapter we will devote our attention on how such models are built in an increasingly complex setting.
We will start with the most simple case derived with restrictive assumptions regarding the time-evolution of the systems under investigation and what impact those assumptions have on the prognosis of tipping events.
We will thus evaluate how, as we subsequently drop some of these assumptions, the models become more realistic at the cost of increased complexity in the analysis.
This construction will aid the intuition on how to mathematically investigate catastrophic events in dynamical systems with the aim of understanding how global properties of the model can be exploited in order to predict them.
% Low-dimensional dynamical systems
\subfile{subsec3.1}
% High-dimensional systems
\subfile{subsec3.2}
% A comprehensive catalogie of spatio-temporal early-warning signals
\subfile{subsec3.3}
\end{document}
