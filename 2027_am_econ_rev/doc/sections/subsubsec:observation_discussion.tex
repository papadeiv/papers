\documentclass[../main.tex]{subfiles}
\begin{document}
\subsubsection{Observations and discussion}\label{subsubsec:observation_discussion}

GW's model introduced in \cite{Galor00} is valuable in that it describes the evolution and development of a complex society in a three-stage process: the system \eqref{eq:galor_system} starts off in a state of Malthusian trap where progress is slow and the economy and population stagnates; this is followed by a transient and highly nonlinear regime of post-Malthusian growth; eventually the economy settles onto a stable equilibrium where the regime is described by constant technological and economic growth and a declined but sustainable level of population.
The purpose of \cite{Lagerlof06} is to provide explicit functional forms to \eqref{eq:galor_system} so that simulations can be performed. 
The objective of said simulations is to enrich the insights of the long-run growth model described qualitatively in \cite{Galor00} with empirical/numerical evidence while providing new results in observed features of the system that were not predicted by the original paper (i.e. the \textit{``oscillatory''} behaviour). 
In this regard the authour names his work a quantitative exercise. The choice of the values of the parameters is not arbitrary but selected to be in line with data.
ICs appear to be a calibration of handpicked values that display the desired result for the long-run growth model.
While Lagerlöf argues that the calibration and simulation pipeline achieves a quantitative description of the model, no attempt is made in characterising it systematically by rigorous analysis of the dynamical system.
Indeed the authour himself observes that the \textit{``oscillaroy''} behaviour disappears from the transient regime as the fixed time cost of children $\tau$ gets sufficiently high ($\tau^{*} = (1-\alpha)/(2-\alpha)$), however no discussion nor mathematical justification is offered for it.
Furthermore there is no indication on wether starting the system from different ICs would have yielded the same long-run results, nor it is expected to be.
In order to fill these gaps we propose a full description of GW's model by analysing its structure in a dynamical systems framework.
With this we wish to fully characterise the various regimes in which the system can settle in the long-run by investigating its phase space and bifurcations. 

\end{document}
