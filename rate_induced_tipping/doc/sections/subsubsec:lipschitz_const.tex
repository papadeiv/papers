\documentclass[../main.tex]{subfiles}
\begin{document}
\subsubsection{Computing Lipschitz constant}\label{subsubsec:lipschitz_const}

We can use Theorem \ref{thm:mean_value_thm} to determine $L$ for a given function $f$.
Since we consider parameter shifts of the form of polynomials of sigmoids we automatically restrict ourselves to functions whose derivatives $\newprime{f}$ are bounded and therefore Lipschitz as per Corollary 2.1.
The algorithm proceeds as follows:
\begin{enumerate}
     \item given $f$ compute $\newprime{f}$;
     \item find the set $\Gamma$ of critical points of $\newprime{f}$, i.e. $\Gamma = \{x\in \mathbb{R}\;:\;\pprime{f}(x)=0\}$;
     \item find the supremum of the set $|\newprime{f}(\Gamma)|$.
\end{enumerate}
In other words
\begin{equation}\label{eq:lipschitz_constant}
        L = \sup\,\{|\newprime{f}(x)|\,,\;\forall x \in \mathbb{R}\;:\;\pprime{f}(x)=0\}\,.
\end{equation}

\end{document}
