\documentclass[../main.tex]{subfiles}
\begin{document}
\subsection{High-dimensional systems}\label{subsec3.2}
The construction of a dynamical model that exhibits tipping points driven by different phenomena has so far relied on a number of assumptions that have been subsequently relaxed in order to achieve a more flexible and realistic mathematical description of observed critical events. 
Throughout the previous analysis we have however restricted ourselves in considering a finite number of observables whose time-evolution, if taken to be a model of real phenomena defined on a spatio-temporal domain, represent the dynamics of spatially-aggregate measures.
In other words we have so far only consider observables that are instrincly defined over a spatial domain whose change with time happens uniformly throughout space. This simplification is powerful, as it allowed the derivation of robust mathematical frameworks in e.g. \cite{Kuehn11} and \cite{Ashwin12} for the analysis of tipping events and their precursors however it also implies the discard of any potential instantaneous change in the spatial distribution of those state variables.
Indeed ignoring how obervables change in space and only considering the time-evolution of their properties automatically restrict the analysis of either events that are not defined on a spatial domain or spatially heterogeneous models that we can only measure in an aggregate sense (i.e. by considering spatial means of the observables).
In both instances, relying on such simplification heavily implies the loss of meaningfull information and methods of mathematical analysis thus preventing the detection of precursors s.a. self-organising pattern-formation.
As such this last strong assumption of spatial homogeneity will be dropped as we will now consider observables $x\mapsto u(x)$, $x\in \Omega\subseteq\mathbb{R}^{d}$ ($d=1,2,3$) as (scalar) functions varying in both space and time.
These functions $u(x,t)\in V\times \mathbb{R}^{+}$ can be thought as solutions of space-time partial differential equations (PDEs) on a functional space $V$ which is infinite dimensional.
We will thus refer to these spatio-temporal dynamical systems, whose phase space coincides with $V$, as high-dimensional as opposed to the low-dimensional case insofar treated for a set of finite observables $x\in \mathbb{R}^{n}$ with $n\ll\infty$.
% Intrinsic discrete phenomena 
\subfile{subsubsec3.2.1}
% Instabilities in reaction-diffusion problems 
\subfile{subsubsec3.2.2}
\end{document}
